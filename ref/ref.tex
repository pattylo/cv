%%%%%%%%%%%%%%%%%%%%%%%%%%%%%%%%%%%%%%%%%%%%%%%%%%%%%%%%%%%%%%%%%%%%%%%%%%%%%%%%
%2345678901234567890123456789012345678901234567890123456789012345678901234567890
%        1         2         3         4         5         6         7         8

\documentclass[letterpaper, 10 pt, conference]{ieeeconf}  % Comment this line out if you need a4paper

%\documentclass[a4paper, 10pt, conference]{ieeeconf}      % Use this line for a4 paper

\IEEEoverridecommandlockouts                              % This command is only needed if 
                                                          % you want to use the \thanks command

\overrideIEEEmargins                                      % Needed to meet printer requirements.

%In case you encounter the following error:
%Error 1010 The PDF file may be corrupt (unable to open PDF file) OR
%Error 1000 An error occurred while parsing a contents stream. Unable to analyze the PDF file.
%This is a known problem with pdfLaTeX conversion filter. The file cannot be opened with acrobat reader
%Please use one of the alternatives below to circumvent this error by uncommenting one or the other
%\pdfobjcompresslevel=0
%\pdfminorversion=4

% See the \addtolength command later in the file to balance the column lengths
% on the last page of the document

% The following packages can be found on http:\\www.ctan.org
%\usepackage{graphics} % for pdf, bitmapped graphics files
%\usepackage{epsfig} % for postscript graphics files
%\usepackage{mathptmx} % assumes new font selection scheme installed
%\usepackage{times} % assumes new font selection scheme installed
%\usepackage{amsmath} % assumes amsmath package installed
%\usepackage{amssymb}  % assumes amsmath package installed
\usepackage{hyperref}
% \usepackage[numbers]{natbib}
\usepackage{mathtools}
\usepackage {amsmath}
\usepackage{amsfonts}
\usepackage{tabularray}

% \usepackage{algorithm}
% \usepackage{algorithmic}
\usepackage[ruled,linesnumbered]{algorithm2e}

% \usepackage[showframe]{geometry}% http://ctan.org/pkg/geometry
\usepackage{lipsum}% http://ctan.org/pkg/lipsum
\usepackage{multicol}% http://ctan.org/pkg/multicols
\usepackage{graphicx}% http://ctan.org/pkg/graphicx



\title{\LARGE \bf
Landing a Quadrotor on a Ground Vehicle without Exteroceptive Airborne Sensors:
A Non-Robocentric Framework and Implementation*
}


\author{Li-Yu Lo$^{1}$ \IEEEmembership{Student Member, IEEE}, Boyang Li$^{2}$, Chih-Yung Wen$^{1}$, and Ching-Wei Chang$^{1\dagger}$ 
% <-this % stops a space
\thanks{*This research is funded by the Innovation and Technology Commission of 
Hong Kong under grant number ITT/027/19GP and PolyU
Presidential PhD Fellowship.}
% <-this % stops a space
\thanks{$^{1}$Li-Yu Lo, Chih-Yung Wen and Ching-Wei Chang
        are with AIRo-LAB, Department of Aeronautical and Aviation Engineering, 
        The Hong Kong Polytechnic University, 
        Kowloon, Hong Kong 
		(e-mail: patty.lo@connect.polyu.hk; chihyung.wen@polyu.edu.hk; chingwei.chang@connect.polyu.hk).
        }
\thanks{$^{2}$Boyang Li is with 
School of Engineering, The University of Newcastle, 
Callaghan, NSW 2308, Australia 
(e-mail: boyang.li@newcastle.edu.au).
    }%
\thanks{$^{\dagger}$ corresponding author
	}%
}


\begin{document}



\maketitle
\thispagestyle{empty}
\pagestyle{empty}


%%%%%%%%%%%%%%%%%%%%%%%%%%%%%%%%%%%%%%%%%%%%%%%%%%%%%%%%%%%%%%%%%%%%%%%%%%%%%%%%
\begin{abstract}

        This research addresses the problem of a quadrotor UAV landing on a ground vehicle. 
        Yet, unlike most existing literature, 
        we transfer most sensing and computing tasks to the ground vehicle,
        designing the landing system in a non-robocentric fashion.
        Such a framework greatly alleviates the payload burden,
        allowing more resource allocation for the quadrotor UAV. 
        To validate the proposed framework, 
        the implementation starts with relative pose estimation
        through detection and tracking of LED markers on an aerial vehicle. 
        The 6 DoF orientation and position information is 
        then returned through a PnP-based algorithm. 
        Successively, by considering the visibility and dynamic constraints
		in the target local frame, 
        the motion planning module computes an optimized landing trajectory, such 
        that the aerial vehicle stays within a safety corridor and performs the landing mission. 
        Through experiments, we demonstrate the 
        applicability of this research work, 
        in which a quadrotor could be guided remotely and landed on a moving ground 
        vehicle smoothly without the support from any airborne exteroceptive  
		sensors and computers. 

\end{abstract}

\section*{Supplementary Material}
\label{sec:SM}
Supporting video of this paper is available at:  \url{https://youtu.be/7wiCh46MQmc} 



%%%%%%%%%%%%%%%%%%%%%%%%%%%%%%%%%%%%%%%%%%%%%%%%%%%%%%%%%%%%%%%%%%%%%%%%%%%%%%%%
\section{INTRODUCTION}
\cite{lo2021dynamic}\cite{lo2024adaptive}\cite{lo2023landing}
	


\addtolength{\textheight}{-0cm}   % This command serves to balance the column lengths
                                  % on the last page of the document manually. It shortens
                                  % the textheight of the last page by a suitable amount.
                                  % This command does not take effect until the next page
                                  % so it should come on the page before the last. Make
                                  % sure that you do not shorten the textheight too much.

%%%%%%%%%%%%%%%%%%%%%%%%%%%%%%%%%%%%%%%%%%%%%%%%%%%%%%%%%%%%%%%%%%%%%%%%%%%%%%%%



%%%%%%%%%%%%%%%%%%%%%%%%%%%%%%%%%%%%%%%%%%%%%%%%%%%%%%%%%%%%%%%%%%%%%%%%%%%%%%%%



%%%%%%%%%%%%%%%%%%%%%%%%%%%%%%%%%%%%%%%%%%%%%%%%%%%%%%%%%%%%%%%%%%%%%%%%%%%%%%%%
% \section*{APPENDIX}

% Appendixes should appear before the acknowledgment.

% \section*{ACKNOWLEDGMENT}

% The preferred spelling of the word �acknowledgment� in America is without an �e� after the �g�. Avoid the stilted expression, �One of us (R. B. G.) thanks . . .�  Instead, try �R. B. G. thanks�. Put sponsor acknowledgments in the unnumbered footnote on the first page.



%%%%%%%%%%%%%%%%%%%%%%%%%%%%%%%%%%%%%%%%%%%%%%%%%%%%%%%%%%%%%%%%%%%%%%%%%%%%%%%%





\bibliographystyle{IEEEtran}
\bibliography{references}



\end{document}
