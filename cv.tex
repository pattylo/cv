%%%%%%%%%%%%%%%%%%%%%%%%%%%%%%%%%%%%%%%%%%%%%%%%%%%%%%%%%%%%%%%%%%%%%%%%%%%%%%%%
% Medium Length Graduate Curriculum Vitae
% LaTeX Template
% Version 1.2 (3/28/15)
%
% This template has been downloaded from:
% http://www.LaTeXTemplates.com
%
% Original author:
% Rensselaer Polytechnic Institute 
% (http://www.rpi.edu/dept/arc/training/latex/resumes/)
%
% Modified by:
% Daniel L Marks <xleafr@gmail.com> 3/28/2015
%
% Important note:
% This template requires the res.cls file to be in the same directory as the
% .tex file. The res.cls file provides the resume style used for structuring the
% document.
%
%%%%%%%%%%%%%%%%%%%%%%%%%%%%%%%%%%%%%%%%%%%%%%%%%%%%%%%%%%%%%%%%%%%%%%%%%%%%%%%%

%-------------------------------------------------------------------------------
%	PACKAGES AND OTHER DOCUMENT CONFIGURATIONS
%-------------------------------------------------------------------------------

%%%%%%%%%%%%%%%%%%%%%%%%%%%%%%%%%%%%%%%%%%%%%%%%%%%%%%%%%%%%%%%%%%%%%%%%%%%%%%%%
% You can have multiple style options the legal options ones are:
%
%   centered:	the name and address are centered at the top of the page 
%				(default)
%
%   line:		the name is the left with a horizontal line then the address to
%				the right
%
%   overlapped:	the section titles overlap the body text (default)
%
%   margin:		the section titles are to the left of the body text
%		
%   11pt:		use 11 point fonts instead of 10 point fonts
%
%   12pt:		use 12 point fonts instead of 10 point fonts
%
%%%%%%%%%%%%%%%%%%%%%%%%%%%%%%%%%%%%%%%%%%%%%%%%%%%%%%%%%%%%%%%%%%%%%%%%%%%%%%%%

\documentclass[overlapped]{res} 

% Default font is the helvetica postscript font
\usepackage{helvet}
\usepackage{url}
\usepackage{hyperref}
\usepackage{xcolor}
\newcommand{\pink}[1]{\textcolor{pink!40!red}{#1}} % Adjust the percentage as needed



% Increase text height
\textheight=728pt

\begin{document}

%-------------------------------------------------------------------------------
%	NAME AND ADDRESS SECTION
%-------------------------------------------------------------------------------
\name{Patrick Li-Yu Lo}

% Note that addresses can be used for other contact information:
% -phone numbers
% -email addresses
% -linked-in profile

\address{Taiwanese Passport Holder\\Reside in Kowloon, Hong Kong}
\address{\hfill liyulo1987@gmail.com \\ \pink{\texttt{\url{https://pattylo.github.io/}}}} 

% Uncomment to add a third address
%\address{Address 3 line 1\\Address 3 line 2\\Address 3 line 3}
%-------------------------------------------------------------------------------

\begin{resume}

\begin{format}
  \title{l}\dates{r}\\
  % \employer{l}\location{r}\\
  \body\\
\end{format}

%-------------------------------------------------------------------------------
%	EDUCATION SECTION
%-------------------------------------------------------------------------------
\section{EDUCATION}
\textbf{The Hong Kong Polytechnic University} \hfill \textit{Sep '22 - Aug '24}\\
{\sl M.Phil. in Robotics \& Control, GPA: 3.77/4.3, Presidential Fellowship Scheme \\ 
Department of Aeronautical and Aviation Engineering}\\ 
\textbf{Thesis}: On Improving the Adaptivity of Controllers and Estimators for Mobile Robots in Challenging Operational Conditions
\\ \\ 
\textbf{The Hong Kong Polytechnic University} \hfill \textit{Sep '17 - Jun '21}\\
{\sl B.Eng. (1st Honour), GPA: 3.75/4.3\\ 
Department of Aeronautical and Aviation Engineering}\\ 
\textbf{Thesis}: Vision-based Navigation of Quadrotor UAV
\\ \\ 
\textbf{University of Queensland, Australia} \hfill \textit{Feb '20 - Jul '20}\\
{\sl Academic Exchange \\ 
Department of Aerospace Engineering}
\par\noindent\hrule width \linewidth % Horizontal line limited to the width of the page

\section{EMPLOYMENT}
\textbf{The Hong Kong University of Science and Technology} \hfill \textit{Jul '21 - Jul '22}\\
{\sl Research Assistant\\
Hong Kong Center for Construction Robotics}\\ 
\textbf{Projects:} CV-based Construction Logistics Monitoring System; Controller Design for Prefabrication Components Installation.\\
\textbf{Duties:} Conducted literature review on adversarial training; performed data augmentation; conducted image pre-processing; co-designed auxiliary hardware setup for better image captures; coded in Arduino \& PyTorch.

\par\noindent\hrule width \linewidth % Horizontal line limited to the width of the page

%-------------------------------------------------------------------------------

%-------------------------------------------------------------------------------
%	PROJECTS SECTION
%-------------------------------------------------------------------------------
\section{RESEARCH PUBLICATIONS}
\begin{itemize}
  \item \underline{\textbf{Lo, L. Y.}}, Y. Hu, B. Li, C.-Y. Wen, and Y. Yang, “An Adaptive Model Predictive Control for Unmanned Underwater Vehicles Subject to External Disturbances and Measurement Noise,” Accepted to 14th IEEE Asian Control Conference (ASCC), 2024. \textit{Links}: \pink{{\texttt{\href{https://pattylo.github.io/assets/pdf/ascc.pdf}{pdf}}} {\texttt{\href{https://github.com/HKPolyU-UAV/bluerov2/tree/eskf_obs2}{code}}}}
  \item \underline{\textbf{Lo, L. Y.}}, B. Li, C.-Y. Wen, and C.-W. Chang, “Experimental Non-Robocentric Dynamic Landing of Quadrotor UAVs with On-Ground Sensor Suite,” Submitted to IEEE Transactions on Instrumentation and Measurement (TIM), 2024. \textit{Links}: \pink{{\texttt{\href{https://pattylo.github.io/assets/pdf/alan_tim.pdf}{pdf}}} {\texttt{\href{https://github.com/HKPolyU-UAV/ALAN}{code}}}} 
  \item \underline{\textbf{Lo, L. Y.}}, B. Li, C.-Y. Wen, and C.-W. Chang, “Landing a Quadrotor on a Ground Vehicle without Exteroceptive Airborne Sensors: A Non-Robocentric Framework and Implementation,” in 2023 IEEE 26th International Conference on Intelligent Transportation Systems (ITSC). IEEE, 2023, pp. 6080–6087. \textit{Links}: \pink{{\texttt{\href{https://pattylo.github.io/assets/pdf/alan_conf_final.pdf}{pdf}}}}
  % \item Jiang, B., Li, B., Zhou, W., \underline{\textbf{Lo, L. Y.}}, Chen, C. K., \& Wen, C. Y. (2022). Neural network based model predictive control for a quadrotor UAV. Aerospace, 9(8), 460. \underline{\texttt{\href{https://pattylo.github.io/}{webpage}}}
  \item \underline{\textbf{Lo, L. Y.}}, C. H. Yiu, Y. Tang, A.-S. Yang, B. Li, and C.-Y. Wen, “Dynamic Object Tracking on Autonomous UAV System for Surveillance Applications,” Sensors, vol. 21, no. 23, p. 7888, 2021. Editor's Choice Article. \textit{Links}: \pink{{\texttt{\href{https://pattylo.github.io/assets/pdf/sensors-21-07888.pdf}{pdf}}} {\texttt{\href{https://github.com/HKPolyU-UAV/auto}{code}}}}
\end{itemize}

\par\noindent\hrule width \linewidth % Horizontal line limited to the width of the page
\section{AWARDS \& SCHOLARSHIP}
\begin{itemize}
  \item First Runner-up of UAV Challenge, 2023 \& 2024 IEEE International Conference on Unmanned Aircraft Systems (ICUAS). \textit{Links}: \pink{
    \texttt{\href{https://www.polyu.edu.hk/publications/pulse-polyu/issue/202308/achievements/aae-team-wins-first-runner-up-prize-at-icuas-23-uav-competition}{web 2023}}
  } 
  \item PolyU Presidential Postgraduate Fellowship Scheme ('22 - '24).
  \item Dean’s list of PolyU Faculty of Engineering: '17/18, '18/19 \& '20/21.
  \item PolyU Undergraduate APEC Entry Scholarship ('17 - '21).
\end{itemize}
\par\noindent\hrule width \linewidth % Horizontal line limited to the width of the page

%-------------------------------------------------------------------------------
\section{SELECTED\\PROJECT}

\textbf{Relative State Estimation for Non-Inertial Control Systems}  
\hfill \textit{Sep '22 - Present} \\
We study the relative state estimation problem for the feedback loop of non-inertial control systems (e.g., ground-centralized UAV-UGV heterogeneous teams) based on visual measurements, control input signals, and observed disturbance. 

\begin{itemize}
  \item \textbf{Main modules:} Investigated the unknown input problem in Kalman filters; designed adaptive extended state observer to extract the real input; fused the lump disturbance and control input into the prediction model of IEKF; performed stability analysis in the sense of Lyapunov for the observer.
  \item \textbf{Tools:} C++/Python in ROS, Gazebo, PX4, Docker.
  % \item \textbf{Achievements:} Paper accepted at \textit{2024 IEEE Asian Control Conference}.
\end{itemize}

\textbf{Observer-based MPC for Unmanned Underwater Vehicle}  
\hfill \textit{Aug '23 - Mar  '24} \\
We designed a novel error-state extended state observer subject to physical sensor models for adaptive nonlinear UUV MPC. 

\begin{itemize}
  \item \textbf{Main modules:} Investigated the IMU model and ESKF; applied Fossen's UUV equation to design the prediction model; integrated the lump disturbance into the prediction model; performed stability analysis in the sense of Lyapunov for both observer and controller.
  \item \textbf{Tools:} C++/Python in ROS, Gazebo, Acados/Casadi, BlueROV2 SITL, Docker.
  \item \textbf{Achievements:} Paper accepted at \textit{2024 IEEE Asian Control Conference}.
\end{itemize}

\textbf{Towards Non-Robocentric Dynamic Landing for Quadrotor UAVs} 
\hfill \textit{Jul '22 - Aug  '23} \\
We proposed a novel sensing configuration for the UAV dynamic landing problem where no airborne exteroceptive sensors were used.

\begin{itemize}
  \item \textbf{Main modules:} Carried out stereo camera image processing; proposed IEKF-based state estimator on SE(3); designed kinematic and dynamic constrained trajectory with differential flatness \& minimized jerk; coded PID outer-loop flight controller; conducted heterogeneous UAV-UGV hardware system design and physical experiments in VICON.
  \item \textbf{Tools:} C++/Python in ROS, Gazebo, OSQP, PX4, Docker.
  \item \textbf{Achievements:} Paper published in \textit{2023 IEEE International Conference on Intelligent Transportation System}; extended version submitted to \textit{IEEE Transactions on Instrumentation and Measurement}; knowledge transferred (partial) to \textit{HKSAR Environmental Protection Department}.
\end{itemize}

% \textbf{CV-based Construction Logistics Monitoring System} 
% \\
% Prototyped an enhanced Automatic License Plate Recognition (ALPR) system for trucks on construction sites which was a collaboration project with the industry.

% \begin{itemize}
%   \item \textbf{Main modules:} Conducted literature review on adversarial training; performed data augmentation; conducted image pre-processing; co-designed auxiliary hardware setup for better image captures.
%   \item \textbf{Tools:} PyTorch, Arduino.
%   \item \textbf{Achievements:} Knowledge transferred to \textit{Hong Kong Sun Hung Kai Properties}.
% \end{itemize}

\textbf{Vision-based Navigation of Quadrotor UAV} 
\hfill \textit{Aug '20 - May  '21} \\
We worked on a fully autonomous UAV with SLAM, dynamic object tracking, path planning, trajectory generation, and controller modules. The team specifically focused on SLAM and dynamic object tracking.

\begin{itemize}
  \item \textbf{Main modules:} Conducted CNN training for object detection; carried out stereo camera image processing; proposed EKF-based state estimator for object tracking; proposed visual-dynamic sensor fusion with TCN/LSTM models to solve the unknown input problem.
  \item \textbf{Tools:} C++/Python in ROS, Gazebo, Darknet, OpenCV DNN, PyTorch, GTSAM, PX4.
  \item \textbf{Achievements:} 2 papers were published in \textit{Sensors} and \textit{Aerospace} as an undergraduate RA; Github repo received $\sim$100 stars.
\end{itemize}
\par\noindent\hrule width \linewidth % Horizontal line limited to the width of the page

%-------------------------------------------------------------------------------
\section{CORE SKILLS \& KNOWLEDGE}
\begin{itemize}
  \item \textbf{State Estimation:} EKF, Graph-SLAM, Extended State Observer, Sensor-Fusion, Stereo Camera, LiDAR, IMU, Lie Theory.
  \item \textbf{Control \& Optimization:} UAV \& UUV Dynamic Analysis, Optimal Control, Nonlinear Control Theory, Trajectory Generation, Convex Optimization, Machine Learning. 
  \item \textbf{Software/Hardware:} ROS in C++/Python, Gazebo, PyTorch, GTSAM, Acados, Docker, PX4 (\& SITL), BlueROV2 (\& SITL).
  \item \textbf{Links:} \pink{{\texttt{\href{https://pattylo.github.io/projects/}{Project Pages}}}, {\texttt{\href{https://github.com/pattylo/learning_jungle}{Learning Notes}}}}.
\end{itemize}
\par\noindent\hrule width \linewidth % Horizontal line limited to the width of the page
%-------------------------------------------------------------------------------

%-------------------------------------------------------------------------------
%	EXPERIENCE SECTION
%-------------------------------------------------------------------------------
% Modify the format of each position


% \employer{PTP Turbo Blankets}
% \location{Austin, Texas}
% \dates{May 2014-Present}
% \title{\textbf{Content Marketing}}
% \begin{position}
% I produce our social media content, including some of our graphics. I develope our content strategies and impliment them upon approval. I maintain our company's online presence and showcase our industry expertise by leveraging our presence in the automotive racing and performance community.
% \end{position}
%-------------------------------------------------------------------------------
\section{OTHERS}
\begin{itemize}
  \item \textbf{Languages}: English (IELTS 7.5), Mandarin (Native), Cantonese (Proficient), Taiwanese (Proficient).
  \item \textbf{Services}: Reviewer of IEEE Transactions on Mechatronics; IEEE Sensor Journal; 2023 IEEE International Conference on Intelligent Transportation System.
  \item \textbf{Volunteering}: Student ambassador of PolyU ('19-'21); Student/Teaching Assistant at HeartFire Education Service, China (Dec '19), African Evangelistic Enterprise, Rwanda (Jun '19), Royal University of Phnom Penh, Cambodia (May - Jun '18).
\end{itemize}


% \employer{The Hong Kong Polytechnic University}
% \location{Hong Kong}
% \dates{Jun '19 - Dec '23}
% \title{\textbf{Teaching Assistant}}
% \begin{position}
% Course: ENG2002 Computer Programming, Sep-Dec '23; AAE1BN01 Introduction to Aviation Industry, Jun-Aug '23; COMP3S02 Socially Responsible Global Leadership in a Digital World (Rwanda), Jun '19.
% \end{position}

% \employer{IEEE}
% \location{Hong Kong}
% \dates{Sep '22 - Present}
% \title{\textbf{Reviewer}}
% \begin{position}
% Journal/Conference: IEEE Transactions on Mechatronics; IEEE Sensor Journal; 2023 IEEE International Conference on Intelligent Transportation System.
% \end{position}

% \dates{May '18 - Jun '21}
% \title{\textbf{Volunteering}}
% \begin{position}
% Events: Student ambassador of PolyU ('19-'21); Volunteer at HeartFire Education Service (Dec '19), African Evangelistic Enterprise (Jun '19), Royal University of Phnom Penh (May - Jun '18).
% \end{position}

% \dates{}
% % \title{\textbf{Others}}
% \title{}
% \begin{position}
% Languages: English (IELTS 7.5), Mandarin (Native), Cantonese (Proficient), Taiwanese (Proficient).\\
% Hobbies: Guitar; Baseball; Skateboarding.
% \end{position}
% \section{SERVICE \& OTHERS}

% \employer{The Hong Kong Polytechnic University}
% \location{Hong Kong}
% \dates{Jun '19 - Dec '23}
% \title{\textbf{Teaching Assistant}}
% \begin{position}
% Course: ENG2002 Computer Programming, Sep-Dec '23; AAE1BN01 Introduction to Aviation Industry, Jun-Aug '23; COMP3S02 Socially Responsible Global Leadership in a Digital World (Rwanda), Jun '19.
% \end{position}

% \employer{IEEE}
% \location{Hong Kong}
% \dates{Sep '22 - Present}
% \title{\textbf{Reviewer}}
% \begin{position}
% Journal/Conference: IEEE Transactions on Mechatronics; IEEE Sensor Journal; 2023 IEEE International Conference on Intelligent Transportation System.
% \end{position}

% \dates{May '18 - Jun '21}
% \title{\textbf{Volunteering}}
% \begin{position}
% Events: Student ambassador of PolyU ('19-'21); Volunteer at HeartFire Education Service (Dec '19), African Evangelistic Enterprise (Jun '19), Royal University of Phnom Penh (May - Jun '18).
% \end{position}

% \dates{}
% \title{\textbf{Others}}
% \begin{position}
% Languages: English (IELTS 7.5), Mandarin (Native), Cantonese (Proficient), Taiwanese (Proficient).\\
% Hobbies: Guitar; Baseball; Skateboarding.
% \end{position}

\end{resume}
\end{document}